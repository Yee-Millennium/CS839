\documentclass[11pt]{article}

\usepackage{preamble}
%%%%% NEW MATH DEFINITIONS %%%%%

\usepackage{amsmath,amsfonts,bm}
% comment out the font for arxiv; iclr is times
% \usepackage{times}
\usepackage{amssymb}
\usepackage{amsthm}
\usepackage{graphicx}
\usepackage{caption}
\usepackage{color}
%\usepackage{media9}
%\usepackage{subfigure}
\usepackage{comment}
\usepackage{enumitem}
\usepackage{mathtools}
\usepackage{thmtools, thm-restate}
% \usepackage[ruled]{algorithm2e}
\usepackage{algorithm}
\usepackage[noend]{algorithmic}
\usepackage{verbatim}


% Custom Macros
\newcommand{\diag}{\operatorname{diag}}
\newcommand{\innp}[1]{\left\langle #1 \right\rangle}
\newcommand{\tildeF}{\widetilde{F}}
\newcommand{\tG}{\widetilde{G}}

\newcommand{\jd}[1]{{\color{purple}{\textbf{JD:} #1}}}
\newcommand{\xc}[1]{{\color{orange}{[\textbf{XC:} #1}]}}
\newcommand{\yw}[1]{{\color{teal}{[\textbf{AA:} #1}]}}
\newcommand{\yee}[1]{{\color{cyan}{[\textbf{YW:} #1}]}}

\newcommand{\DATE}[1]{{\color{purple}{[\textbf{Date:} #1}]}}



\iffalse
\theoremstyle{plain} \numberwithin{equation}{section}
\newtheorem{theorem}{Theorem}[section]
\numberwithin{theorem}{section}
\newtheorem{corollary}[theorem]{Corollary}
\newtheorem{conjecture}{Conjecture}
\newtheorem{lemma}[theorem]{Lemma}
\newtheorem{proposition}[theorem]{Proposition}
\newtheorem{claim}[theorem]{Claim}
\newtheorem{fact}[theorem]{Fact}
\theoremstyle{definition}
\newtheorem{definition}[theorem]{Definition}
\newtheorem{finalremark}[theorem]{Final Remark}
\newtheorem{remark}[theorem]{Remark}
\newtheorem{example}[theorem]{Example}
\newtheorem{observation}[theorem]{Observation}
\newtheorem{assumption}{Assumption}
\fi


% Mark sections of captions for referring to divisions of figures
\newcommand{\figleft}{{\em (Left)}}
\newcommand{\figcenter}{{\em (Center)}}
\newcommand{\figright}{{\em (Right)}}
\newcommand{\figtop}{{\em (Top)}}
\newcommand{\figbottom}{{\em (Bottom)}}
\newcommand{\captiona}{{\em (a)}}
\newcommand{\captionb}{{\em (b)}}
\newcommand{\captionc}{{\em (c)}}
\newcommand{\captiond}{{\em (d)}}

% Highlight a newly defined term
\newcommand{\newterm}[1]{{\bf #1}}


% Figure reference, lower-case.
\def\figref#1{figure~\ref{#1}}
% Figure reference, capital. For start of sentence
\def\Figref#1{Figure~\ref{#1}}
\def\twofigref#1#2{figures \ref{#1} and \ref{#2}}
\def\quadfigref#1#2#3#4{figures \ref{#1}, \ref{#2}, \ref{#3} and \ref{#4}}
% Section reference, lower-case.
\def\secref#1{section~\ref{#1}}
% Section reference, capital.
\def\Secref#1{Section~\ref{#1}}
% Reference to two sections.
\def\twosecrefs#1#2{sections \ref{#1} and \ref{#2}}
% Reference to three sections.
\def\secrefs#1#2#3{sections \ref{#1}, \ref{#2} and \ref{#3}}
% Reference to an equation, lower-case.
% \def\eqref#1{equation~\ref{#1}}
% Reference to an equation, upper case
% \def\Eqref#1{Equation~\ref{#1}}
% A raw reference to an equation---avoid using if possible
\def\plaineqref#1{\ref{#1}}
% Reference to a chapter, lower-case.
\def\chapref#1{chapter~\ref{#1}}
% Reference to an equation, upper case.
\def\Chapref#1{Chapter~\ref{#1}}
% Reference to a range of chapters
\def\rangechapref#1#2{chapters\ref{#1}--\ref{#2}}
% Reference to an algorithm, lower-case.
\def\algref#1{algorithm~\ref{#1}}
% Reference to an algorithm, upper case.
\def\Algref#1{Algorithm~\ref{#1}}
\def\twoalgref#1#2{algorithms \ref{#1} and \ref{#2}}
\def\Twoalgref#1#2{Algorithms \ref{#1} and \ref{#2}}
% Reference to a part, lower case
\def\partref#1{part~\ref{#1}}
% Reference to a part, upper case
\def\Partref#1{Part~\ref{#1}}
\def\twopartref#1#2{parts \ref{#1} and \ref{#2}}

\def\ceil#1{\lceil #1 \rceil}
\def\floor#1{\lfloor #1 \rfloor}
\def\1{\bm{1}}
\newcommand{\train}{\mathcal{D}}
\newcommand{\valid}{\mathcal{D_{\mathrm{valid}}}}
\newcommand{\test}{\mathcal{D_{\mathrm{test}}}}

\def\eps{{\epsilon}}


% Random variables
\def\reta{{\textnormal{$\eta$}}}
\def\ra{{\textnormal{a}}}
\def\rb{{\textnormal{b}}}
\def\rc{{\textnormal{c}}}
\def\rd{{\textnormal{d}}}
\def\re{{\textnormal{e}}}
\def\rf{{\textnormal{f}}}
\def\rg{{\textnormal{g}}}
\def\rh{{\textnormal{h}}}
\def\ri{{\textnormal{i}}}
\def\rj{{\textnormal{j}}}
\def\rk{{\textnormal{k}}}
\def\rl{{\textnormal{l}}}
% rm is already a command, just don't name any random variables m
\def\rn{{\textnormal{n}}}
\def\ro{{\textnormal{o}}}
\def\rp{{\textnormal{p}}}
\def\rq{{\textnormal{q}}}
\def\rr{{\textnormal{r}}}
\def\rs{{\textnormal{s}}}
\def\rt{{\textnormal{t}}}
\def\ru{{\textnormal{u}}}
\def\rv{{\textnormal{v}}}
\def\rw{{\textnormal{w}}}
\def\rx{{\textnormal{x}}}
\def\ry{{\textnormal{y}}}
\def\rz{{\textnormal{z}}}

% Random vectors
\def\rvepsilon{{\mathbf{\epsilon}}}
\def\rvtheta{{\mathbf{\theta}}}
\def\rva{{\mathbf{a}}}
\def\rvb{{\mathbf{b}}}
\def\rvc{{\mathbf{c}}}
\def\rvd{{\mathbf{d}}}
\def\rve{{\mathbf{e}}}
\def\rvf{{\mathbf{f}}}
\def\rvg{{\mathbf{g}}}
\def\rvh{{\mathbf{h}}}
\def\rvu{{\mathbf{i}}}
\def\rvj{{\mathbf{j}}}
\def\rvk{{\mathbf{k}}}
\def\rvl{{\mathbf{l}}}
\def\rvm{{\mathbf{m}}}
\def\rvn{{\mathbf{n}}}
\def\rvo{{\mathbf{o}}}
\def\rvp{{\mathbf{p}}}
\def\rvq{{\mathbf{q}}}
\def\rvr{{\mathbf{r}}}
\def\rvs{{\mathbf{s}}}
\def\rvt{{\mathbf{t}}}
\def\rvu{{\mathbf{u}}}
\def\rvv{{\mathbf{v}}}
\def\rvw{{\mathbf{w}}}
\def\rvx{{\mathbf{x}}}
\def\rvy{{\mathbf{y}}}
\def\rvz{{\mathbf{z}}}

% Elements of random vectors
\def\erva{{\textnormal{a}}}
\def\ervb{{\textnormal{b}}}
\def\ervc{{\textnormal{c}}}
\def\ervd{{\textnormal{d}}}
\def\erve{{\textnormal{e}}}
\def\ervf{{\textnormal{f}}}
\def\ervg{{\textnormal{g}}}
\def\ervh{{\textnormal{h}}}
\def\ervi{{\textnormal{i}}}
\def\ervj{{\textnormal{j}}}
\def\ervk{{\textnormal{k}}}
\def\ervl{{\textnormal{l}}}
\def\ervm{{\textnormal{m}}}
\def\ervn{{\textnormal{n}}}
\def\ervo{{\textnormal{o}}}
\def\ervp{{\textnormal{p}}}
\def\ervq{{\textnormal{q}}}
\def\ervr{{\textnormal{r}}}
\def\ervs{{\textnormal{s}}}
\def\ervt{{\textnormal{t}}}
\def\ervu{{\textnormal{u}}}
\def\ervv{{\textnormal{v}}}
\def\ervw{{\textnormal{w}}}
\def\ervx{{\textnormal{x}}}
\def\ervy{{\textnormal{y}}}
\def\ervz{{\textnormal{z}}}

% Random matrices
\def\rmA{{\mathbf{A}}}
\def\rmB{{\mathbf{B}}}
\def\rmC{{\mathbf{C}}}
\def\rmD{{\mathbf{D}}}
\def\rmE{{\mathbf{E}}}
\def\rmF{{\mathbf{F}}}
\def\rmG{{\mathbf{G}}}
\def\rmH{{\mathbf{H}}}
\def\rmI{{\mathbf{I}}}
\def\rmJ{{\mathbf{J}}}
\def\rmK{{\mathbf{K}}}
\def\rmL{{\mathbf{L}}}
\def\rmM{{\mathbf{M}}}
\def\rmN{{\mathbf{N}}}
\def\rmO{{\mathbf{O}}}
\def\rmP{{\mathbf{P}}}
\def\rmQ{{\mathbf{Q}}}
\def\rmR{{\mathbf{R}}}
\def\rmS{{\mathbf{S}}}
\def\rmT{{\mathbf{T}}}
\def\rmU{{\mathbf{U}}}
\def\rmV{{\mathbf{V}}}
\def\rmW{{\mathbf{W}}}
\def\rmX{{\mathbf{X}}}
\def\rmY{{\mathbf{Y}}}
\def\rmZ{{\mathbf{Z}}}

% Elements of random matrices
\def\ermA{{\textnormal{A}}}
\def\ermB{{\textnormal{B}}}
\def\ermC{{\textnormal{C}}}
\def\ermD{{\textnormal{D}}}
\def\ermE{{\textnormal{E}}}
\def\ermF{{\textnormal{F}}}
\def\ermG{{\textnormal{G}}}
\def\ermH{{\textnormal{H}}}
\def\ermI{{\textnormal{I}}}
\def\ermJ{{\textnormal{J}}}
\def\ermK{{\textnormal{K}}}
\def\ermL{{\textnormal{L}}}
\def\ermM{{\textnormal{M}}}
\def\ermN{{\textnormal{N}}}
\def\ermO{{\textnormal{O}}}
\def\ermP{{\textnormal{P}}}
\def\ermQ{{\textnormal{Q}}}
\def\ermR{{\textnormal{R}}}
\def\ermS{{\textnormal{S}}}
\def\ermT{{\textnormal{T}}}
\def\ermU{{\textnormal{U}}}
\def\ermV{{\textnormal{V}}}
\def\ermW{{\textnormal{W}}}
\def\ermX{{\textnormal{X}}}
\def\ermY{{\textnormal{Y}}}
\def\ermZ{{\textnormal{Z}}}

% Vectors
\def\vzero{{\bm{0}}}
\def\vone{{\bm{1}}}
\def\vmu{{\bm{\mu}}}
\def\vtheta{{\bm{\theta}}}
\def\va{{\bm{a}}}
\def\vb{{\bm{b}}}
\def\vc{{\bm{c}}}
\def\vd{{\bm{d}}}
% \def\ve{{\bm{e}}}
\def\ve{{\mathbf{e}}}
\def\vf{{\bm{f}}}
% \def\vg{{\bm{g}}}
\def\vg{{\mathbf{g}}}
\def\vh{{\bm{h}}}
\def\vi{{\bm{i}}}
\def\vj{{\bm{j}}}
\def\vk{{\bm{k}}}
\def\vl{{\bm{l}}}
\def\vm{{\bm{m}}}
\def\vn{{\bm{n}}}
\def\vo{{\bm{o}}}
\def\vp{{\bm{p}}}
\def\vq{{\bm{q}}}
\def\vr{{\bm{r}}}
\def\vs{{\bm{s}}}
\def\vt{{\bm{t}}}
% \def\vu{{\bm{u}}}
\def\vu{{\mathbf{u}}}
% \def\vv{{\bm{v}}}
\def\vv{{\mathbf{v}}}
% \def\vw{{\bm{w}}}
\def\vw{{\mathbf{w}}}
% \def\vx{{\bm{x}}}
\def\vx{{\mathbf{x}}}
% \def\vy{{\bm{y}}}
\def\vy{{\mathbf{y}}}
\def\vz{{\bm{z}}}
\def\dist{{\mathrm{dist}}}

% Elements of vectors
\def\evalpha{{\alpha}}
\def\evbeta{{\beta}}
\def\evepsilon{{\epsilon}}
\def\evlambda{{\lambda}}
\def\evomega{{\omega}}
\def\evmu{{\mu}}
\def\evpsi{{\psi}}
\def\evsigma{{\sigma}}
\def\evtheta{{\theta}}
\def\eva{{a}}
\def\evb{{b}}
\def\evc{{c}}
\def\evd{{d}}
\def\eve{{e}}
\def\evf{{f}}
\def\evg{{g}}
\def\evh{{h}}
\def\evi{{i}}
\def\evj{{j}}
\def\evk{{k}}
\def\evl{{l}}
\def\evm{{m}}
\def\evn{{n}}
\def\evo{{o}}
\def\evp{{p}}
\def\evq{{q}}
\def\evr{{r}}
\def\evs{{s}}
\def\evt{{t}}
\def\evu{{u}}
\def\evv{{v}}
\def\evw{{w}}
\def\evx{{x}}
\def\evy{{y}}
\def\evz{{z}}

% Matrix
\def\mA{{\bm{A}}}
\def\mB{{\bm{B}}}
\def\mC{{\bm{C}}}
\def\mD{{\bm{D}}}
\def\mE{{\bm{E}}}
\def\mF{{\bm{F}}}
\def\mG{{\bm{G}}}
\def\mH{{\bm{H}}}
\def\mI{{\bm{I}}}
\def\mJ{{\bm{J}}}
\def\mK{{\bm{K}}}
\def\mL{{\bm{L}}}
\def\mM{{\bm{M}}}
\def\mN{{\bm{N}}}
\def\mO{{\bm{O}}}
\def\mP{{\bm{P}}}
\def\mQ{{\bm{Q}}}
\def\mR{{\bm{R}}}
\def\mS{{\bm{S}}}
\def\mT{{\bm{T}}}
\def\mU{{\bm{U}}}
\def\mV{{\bm{V}}}
\def\mW{{\bm{W}}}
\def\mX{{\bm{X}}}
\def\mY{{\bm{Y}}}
\def\mZ{{\bm{Z}}}
\def\mBeta{{\bm{\beta}}}
\def\mPhi{{\bm{\Phi}}}
\def\mLambda{{\bm{\Lambda}}}
\def\mSigma{{\bm{\Sigma}}}

% Tensor
\DeclareMathAlphabet{\mathsfit}{\encodingdefault}{\sfdefault}{m}{sl}
\SetMathAlphabet{\mathsfit}{bold}{\encodingdefault}{\sfdefault}{bx}{n}
\newcommand{\tens}[1]{\bm{\mathsfit{#1}}}
\def\tA{{\tens{A}}}
\def\tB{{\tens{B}}}
\def\tC{{\tens{C}}}
\def\tD{{\tens{D}}}
\def\tE{{\tens{E}}}
\def\tF{{\tens{F}}}
\def\tG{{\tens{G}}}
\def\tH{{\tens{H}}}
\def\tI{{\tens{I}}}
\def\tJ{{\tens{J}}}
\def\tK{{\tens{K}}}
\def\tL{{\tens{L}}}
\def\tM{{\tens{M}}}
\def\tN{{\tens{N}}}
\def\tO{{\tens{O}}}
\def\tP{{\tens{P}}}
\def\tQ{{\tens{Q}}}
\def\tR{{\tens{R}}}
\def\tS{{\tens{S}}}
\def\tT{{\tens{T}}}
\def\tU{{\tens{U}}}
\def\tV{{\tens{V}}}
\def\tW{{\tens{W}}}
\def\tX{{\tens{X}}}
\def\tY{{\tens{Y}}}
\def\tZ{{\tens{Z}}}


% Graph
\def\gA{{\mathcal{A}}}
\def\gB{{\mathcal{B}}}
\def\gC{{\mathcal{C}}}
\def\gD{{\mathcal{D}}}
\def\gE{{\mathcal{E}}}
\def\gF{{\mathcal{F}}}
\def\gG{{\mathcal{G}}}
\def\gH{{\mathcal{H}}}
\def\gI{{\mathcal{I}}}
\def\gJ{{\mathcal{J}}}
\def\gK{{\mathcal{K}}}
\def\gL{{\mathcal{L}}}
\def\gM{{\mathcal{M}}}
\def\gN{{\mathcal{N}}}
\def\gO{{\mathcal{O}}}
\def\gP{{\mathcal{P}}}
\def\gQ{{\mathcal{Q}}}
\def\gR{{\mathcal{R}}}
\def\gS{{\mathcal{S}}}
\def\gT{{\mathcal{T}}}
\def\gU{{\mathcal{U}}}
\def\gV{{\mathcal{V}}}
\def\gW{{\mathcal{W}}}
\def\gX{{\mathcal{X}}}
\def\gY{{\mathcal{Y}}}
\def\gZ{{\mathcal{Z}}}


% Sets
\def\sA{{\mathbb{A}}}
\def\sB{{\mathbb{B}}}
\def\sC{{\mathbb{C}}}
\def\sD{{\mathbb{D}}}
% Don't use a set called E, because this would be the same as our symbol
% for expectation.
\def\sF{{\mathbb{F}}}
\def\sG{{\mathbb{G}}}
\def\sH{{\mathbb{H}}}
\def\sI{{\mathbb{I}}}
\def\sJ{{\mathbb{J}}}
\def\sK{{\mathbb{K}}}
\def\sL{{\mathbb{L}}}
\def\sM{{\mathbb{M}}}
\def\sN{{\mathbb{N}}}
\def\sO{{\mathbb{O}}}
\def\sP{{\mathbb{P}}}
\def\sQ{{\mathbb{Q}}}
\def\sR{{\mathbb{R}}}
\def\sS{{\mathbb{S}}}
\def\sT{{\mathbb{T}}}
\def\sU{{\mathbb{U}}}
\def\sV{{\mathbb{V}}}
\def\sW{{\mathbb{W}}}
\def\sX{{\mathbb{X}}}
\def\sY{{\mathbb{Y}}}
\def\sZ{{\mathbb{Z}}}

% Entries of a matrix
\def\emLambda{{\Lambda}}
\def\emA{{A}}
\def\emB{{B}}
\def\emC{{C}}
\def\emD{{D}}
\def\emE{{E}}
\def\emF{{F}}
\def\emG{{G}}
\def\emH{{H}}
\def\emI{{I}}
\def\emJ{{J}}
\def\emK{{K}}
\def\emL{{L}}
\def\emM{{M}}
\def\emN{{N}}
\def\emO{{O}}
\def\emP{{P}}
\def\emQ{{Q}}
\def\emR{{R}}
\def\emS{{S}}
\def\emT{{T}}
\def\emU{{U}}
\def\emV{{V}}
\def\emW{{W}}
\def\emX{{X}}
\def\emY{{Y}}
\def\emZ{{Z}}
\def\emSigma{{\Sigma}}

% entries of a tensor
% Same font as tensor, without \bm wrapper
\newcommand{\etens}[1]{\mathsfit{#1}}
\def\etLambda{{\etens{\Lambda}}}
\def\etA{{\etens{A}}}
\def\etB{{\etens{B}}}
\def\etC{{\etens{C}}}
\def\etD{{\etens{D}}}
\def\etE{{\etens{E}}}
\def\etF{{\etens{F}}}
\def\etG{{\etens{G}}}
\def\etH{{\etens{H}}}
\def\etI{{\etens{I}}}
\def\etJ{{\etens{J}}}
\def\etK{{\etens{K}}}
\def\etL{{\etens{L}}}
\def\etM{{\etens{M}}}
\def\etN{{\etens{N}}}
\def\etO{{\etens{O}}}
\def\etP{{\etens{P}}}
\def\etQ{{\etens{Q}}}
\def\etR{{\etens{R}}}
\def\etS{{\etens{S}}}
\def\etT{{\etens{T}}}
\def\etU{{\etens{U}}}
\def\etV{{\etens{V}}}
\def\etW{{\etens{W}}}
\def\etX{{\etens{X}}}
\def\etY{{\etens{Y}}}
\def\etZ{{\etens{Z}}}

%%% Distribution
% The true underlying data generating distribution
\newcommand{\pdata}{p_{\rm{data}}}

% The empirical distribution defined by the training set
\newcommand{\ptrain}{\hat{p}_{\rm{data}}}
\newcommand{\Ptrain}{\hat{P}_{\rm{data}}}

% The model distribution
\newcommand{\pmodel}{p_{\rm{model}}}
\newcommand{\Pmodel}{P_{\rm{model}}}
\newcommand{\ptildemodel}{\tilde{p}_{\rm{model}}}

% Stochastic autoencoder distributions
\newcommand{\pencode}{p_{\rm{encoder}}}
\newcommand{\pdecode}{p_{\rm{decoder}}}
\newcommand{\precons}{p_{\rm{reconstruct}}}

% Laplace distribution
\newcommand{\laplace}{\mathrm{Laplace}} 

% Specific letter fonts
\newcommand{\E}{\mathbb{E}}
\newcommand{\Ls}{\mathcal{L}}
\newcommand{\R}{\mathbb{R}}
\newcommand{\emp}{\tilde{p}}
\newcommand{\lr}{\alpha}
\newcommand{\reg}{\lambda}
\newcommand{\rect}{\mathrm{rectifier}}
\newcommand{\softmax}{\mathrm{softmax}}
\newcommand{\sigmoid}{\sigma}
\newcommand{\softplus}{\zeta}
\newcommand{\KL}{D_{\mathrm{KL}}}
\newcommand{\Var}{\mathrm{Var}}
\newcommand{\standarderror}{\mathrm{SE}}
\newcommand{\Cov}{\mathrm{Cov}}
\newcommand{\Rho}{\mathrm{P}}

% Wolfram Mathworld says $L^2$ is for function spaces and $\ell^2$ is for vectors
% But then they seem to use $L^2$ for vectors throughout the site, and so does
% wikipedia.
\newcommand{\normlzero}{L^0}
\newcommand{\normlone}{L^1}
\newcommand{\normltwo}{L^2}
\newcommand{\normlp}{L^p}
\newcommand{\normmax}{L^\infty}

\newcommand{\parents}{Pa} % See usage in notation.tex. Chosen to match Daphne's book.

% \DeclareMathOperator*{\argmax}{arg\,max}
% \DeclareMathOperator*{\argmin}{arg\,min}
\DeclareMathOperator*{\argmax}{argmax}
\DeclareMathOperator*{\argmin}{argmin}

\DeclareMathOperator{\sign}{sign}
\DeclareMathOperator{\Tr}{Tr}
\let\ab\allowbreak

\newcommand\independent{\protect\mathpalette{\protect\independenT}{\perp}}
\def\independenT#1#2{\mathrel{\rlap{$#1#2$}\mkern2mu{#1#2}}}


\let\underbrace\LaTeXunderbrace
\let\overbrace\LaTeXoverbrace

\title{Notes of CS 839: Advanced Nonlinear Optimization\\Instructor: Jelena Diakonikolas}
\author{YI WEI}
\date{Sep 2024}


\begin{document}

\maketitle
\tableofcontents

\section{Vector Space}

\yee{TODO: Notes of Sep 4.}

\DATE{Sep 6, 2024}
\begin{example}
    \begin{enumerate}
        \item Induced matrix norms 
                $A \in \R^{m \times n} $
                Let $\| \cdot \|_{a} $ be any norm in $\mathbb{R}^n, \| \cdot  \| _{b}$ be any norm in $R^m$,
                $\| A \|_{a,b} = \max_{x \in \mathbb{R}^n: \| x\|_{a} \le 1 } \| Ax \|_{b} $ \\
                In particular, if $\| \cdot \|_{a} $ and $\| \cdot \|_{b}$ are $l_{p}$ norms:
                \begin{enumerate}
                    \item $a=b=2 \to $ operator/spectral norm
                    \item $a=b=1$:
                            \begin{align}
                                \| A \|_{1,1} &= \max_{x \in \mathbb{R}^n, \| x \|_{1} \le 1} \| Ax \|_{1} \\
                                    &= \max_{1 \le j \le n}\sum_{i=1}^{n}| A_{ij} |
                            \end{align}
                            It's called "max abs column sum"
                    \item $a=b=\infty$:
                            \begin{align*}
                                \| A \| _{\infty,\infty} = \max_{x \in \mathbb{R}^n, \| x \| _{\infty} \le 1}
                                        \| Ax \| _{\infty} = \max_{1 \le i \le m} \sum_{j=1}^{n}|A_{ij}|
                            \end{align*}
                            It's called "max abs row sum norm". 
                    \item $a=1, b=\infty$:
                            \begin{align*}
                                \| A \| _{1,\infty} = \max_{x\in \mathbb{R}^n, \| x \| _{1} \le 1} \| Ax \|_{\infty}
                                        = \max_{1 \le i \le m, 1 \le j \le n} |A_{ij}|
                            \end{align*}
                            where $\| Ax \| _{\infty} = \begin{bmatrix} A_{1}x \\ A_{2}x \\ \vdots \\ A_{n}x \end{bmatrix}$
                \end{enumerate}
    \end{enumerate}
\end{example}

\subsection{Cartesian Product of Vector Space}
Given $m \ge 2$ vector spaces $\mathbb{E}_1, \ldots , \mathbb{E}_{m}$ equipped w/ inner products 
$\langle \cdot, \cdot  \rangle, \ldots , \langle \cdot, \cdot \rangle$, their Cartesian product is the vector
space $\mathbb{E} = \mathbb{E}_1 \times \cdots \times \mathbb{E}_{n}$ containing all m-tuples $( \vv_1, \ldots ,\vv_m ) $ for which basic operations are defined as:
\begin{enumerate}
    \item Addition: $( \vv_{1}, \ldots ,\vv_m ) + ( \vw_1, \ldots ,\vw_{m} ) $ = 
    \item Scaler multiplication: $\alpha \in \mathbb{R}, \alpha (\vv_{1}, \ldots ,\vv_{m}) = 
            (\alpha \vv_{1}, \ldots , \alpha \vv_{m})$
\end{enumerate}
The inner product on $\mathbb{E}$ is defined by:
\begin{align*}
    \langle (\vv_{1}, \ldots ,\vv_{m}), (\vw_{1}, \ldots ,\vw_{m}) \rangle = \sum_{i=1}^{m}\langle v_{i}, w_{i} \rangle _{\mathbb{E}_{i}}
\end{align*}

If $\mathbb{E}_{i}, i \in \{ 1, \ldots ,m \} $ are endowed w/ norms $\|  \| _{E_{i}}$ there a different 
ways of choosing a norm on $\mathbb{E}$
\begin{example}
    \item $\| (\vv_{1}, \ldots ,\vv_{m}) \| = (\sum_{i=1}^{m}\| v_{i} \|_{\mathbb{E}_{i}}^p)^{\frac{1}{p}} $
    \item $\| (\vv_1, \ldots ,\vv_{m}) \| = ( \sum_{i=1}^{m} w_{i}\| v_{i} \|_{\mathbb{E}_{i}}^2  ) $
\end{example}


\subsection{Linear Transformation}
\begin{definition}
    Given two vector spaces $\mathbb{E}$, $\mathbb{V}$, $f: \mathbb{E} \to \mathbb{V}$ is a linear transformation if
    \begin{align*}
        \forall x,y \in \mathbb{E}, \forall \alpha, \beta \in \mathbb{R}:\\
        A(\alpha x + \beta y) = \alpha A(x) + \beta A(y)
    \end{align*}
\end{definition}

\begin{example}
    \begin{enumerate}
        \item All linear transformations from $\mathbb{R}^n \to \mathbb{R}^m$ are of the from
            \begin{align*}
                A(x) = Ax \quad \text{for some matrix } A \in \mathbb{R}^{m \times n}
            \end{align*}
        \item All linear transformations from $\mathbb{R}^{n \times n} \to \mathbb{R}^k$ are of the form:
            \begin{align*}
                A(X) = \begin{bmatrix} \text{trace}(A_{1}^\top X) \\ \text{trace}(A_{2}^\top X) \\ 
                                \vdots \\ \text{trace}(A_{n}^\top X) \end{bmatrix} \quad \forall \; X \in \mathbb{R}^{m \times n}
            \end{align*}
            some matrices $A_{1}, \ldots ,A_{k} \in \mathbb{R}^{m \times n}$
        \item The identity transformation $\mathcal{I}: \mathbb{E} \to \mathbb{E}$ is defined by $\mathcal{I}(x) = x$
    \end{enumerate}
\end{example}

\subsection{The Dual Space}
\begin{definition}
    The dual space of a vector space $\mathbb{E}$ is the space of all linear functionals on $\mathbb{E}$
\end{definition}

For inner product spaces, (Riez Representation) 
for any linear functional $f$, $\exists v \in \mathbb{E}$ s.t $f(x) = \langle \vv,\vx \rangle
    \quad \forall \vx \in \mathbb{E}$. \\
We write $\vv \in \mathbb{E}^*$ (notation). \\
Elements of $\mathbb{E}^*$ and $\mathbb{E}$ are the same if $\mathbb{E}$ we use a norm $\| \cdot \| $,
then in $\mathbb{E}^*$ we use the norm dual to it, defined by (dual norm )
\begin{align*}
    \forall \vy \in \mathbb{E}^*: \| \vy \|_{*} := \max_{\vx \in \mathbb{E}: \| x \| \le 1 } \langle \vy,\vx \rangle
\end{align*}


\begin{theorem}
    Generalized Cauchy-Schwarz:
    \begin{align*}
        \forall \vx \in \mathbb{E}, \forall \vy \in \mathbb{E}^*:
            \| \langle \vx,\vy \rangle \| \le \| \vx \| \| \vy \| _{*}
    \end{align*}
\end{theorem}


\begin{theorem}
    Euclidean norms are self-dual. We say that Euclidean space "self-dual" and write $\mathbb{E} = \mathbb{E}^*$
\end{theorem}

\begin{example}
    \begin{enumerate}
        \item In $\mathbb{R}^d$, with $\langle \vx, \vy \rangle = \vx^\top \vy$
            \begin{enumerate}
                \item The norm dual to $l_{p}$ norm for $p > 1$ is the norm $l_{p}^*$ where 
                    $\frac{1}{p} + \frac{1}{p^*} = 1$. $l_1$ and $l_{\infty}$ are dual to each other.
                \item The norm dual to $\| \cdot \|_{Q}$ for $Q$ symmetric, positive definite is 
                        $\| \cdot \|_{Q^{-1}}$
                        \begin{align*}
                            \| \vx \|_{Q^{-1}} = \Big(\vx^\top Q^{-1} x \Big)^{\frac{1}{2}}
                        \end{align*}
                        If $Q = \text{diag}(w_{1}, \ldots ,w_{d})$ for positive $w_{1}, \ldots ,w_{d}$,
                        then $\| \vx \|_{Q^{-1}} = \Big(\sum_{i=1}^{d} \frac{1}{w_{i}}\vx_{i}^2\Big)^{\frac{1}{2}}$
            \end{enumerate}
        \item $E = E_1 \times \cdots \times E_m$, with $\|  \cdot \|_{E_1}, \ldots , \|  \cdot \| _{E_{m}}$
            \begin{align*}
                \| (\vv_{1}, \ldots ,\vv_{m}) \|_{\mathbb{E}} = \Big(\sum_{i=1}^{m} w_{i} \| \vv_{i} \|_{\mathbb{E}_{i}}^2 \Big)^{\frac{1}{2}}\\
                \| (\vw_{1}, \ldots ,\vw_{m}) \|_{\mathbb{E}^*} = \Big(\sum_{i=1}^{m} \frac{1}{w_{i}} \| \vu_{i} \|_{\mathbb{E}_{i}^*}^2 \Big)^{\frac{1}{2}}
            \end{align*}
    \end{enumerate}
\end{example}

\begin{theorem}
    Bidual space = dual space to $\mathbb{E}^*$. \\
    In finite vector space, $\mathbb{E}^{**} = \mathbb{E}$
\end{theorem}

\begin{theorem}
    $\langle A\vx,\vy \rangle \le \| A \|_{a,b} \| \vx \|_{a} \| \vy \|_{b}$ if $\| \cdot  \|_{a} $
        and $\| \cdot  \| _{b}$ are dual to each other.
\end{theorem}


\subsection{Adjoint Transformation}
\begin{definition}
    Given vector space $\mathbb{E}$ and $\mathbb{V}$, and a linear transformation $A: \mathbb{E} \to \mathbb{V}$,
        the adjoint transformation $A^\top: \mathbb{V}^* \to \mathbb{E}^*$ is defined by
        \begin{align*}
            \langle \vy, A(x) \rangle = \langle A^\top(y), \vx \rangle
        \end{align*}
\end{definition}


\begin{example}
    In particular,
    \begin{enumerate}
        \item If $\mathbb{E} = \mathbb{R}^n, \mathbb{V} = \mathbb{R}^m$, $\langle \vx,\vy \rangle = \vx^\top \vy$,
            then, $A(x) = Ax$ for some $A \in \mathbb{R}^{m \times n}$ and $A^\top (y) = A^\top \vy$
        \item $\mathbb{E} = \mathbb{R}^{m \times n}, \mathbb{V} = \mathbb{R}^k$
    \end{enumerate}
\end{example}

\DATE{Sep 13, 2024}
Given $A: \mathbb{E} \to \mathbb{V}$, $\| \cdot  \|_{\mathbb{E}}, \| \cdot  \|_{\mathbb{E}} $, we define the norm
$\| A \| =\sup_{x\in \mathbb{E}, \| x \|_{\mathbb{E}} \le 1 } \| A(x) \|_{\mathbb{V}} $


\section{Extended Real-Valued Functions}
\begin{definition}
    functions that map some real vector space $(\mathbb{E}, \langle \cdot ,\cdot  \rangle), \| \cdot  \| $ to
    the extended real line -either $\mathbb{R} \bigcup \{ -\infty,+\infty \} \equiv [-\infty,+\infty]$ or 
    $\mathbb{R} \bigcup \{ +\infty \} \equiv (-\infty,+\infty]$
\end{definition}

\begin{align*}
    \begin{aligned}
        \min_{x \in \mathbb{E}} &\quad f(x)
    \end{aligned}
\end{align*}

Consider this problem, why do we even want to include $+\infty$
\begin{enumerate}
    \item $f$ is not everywhere defined  on $\mathbb{E}$, I can assign it to $+\infty$ at points where it's not
    defined. So when it becomes well-defined on all $\mathbb{E}$.
    
    Here we define the domain $=$ effective domain:
    \begin{align*}
        dom(f) = \{ x \in \mathbb{E}: f(x) < +\infty \}
    \end{align*}
    \item We can think of all optimization problems whether constrained or unconstrained, as unconstrained optimization
    problem.
    \begin{align*}
        \begin{aligned}
            \min_{x\in \mathcal{X}}  f(x) \iff \min_{x \in \mathbb{E}} f(x) + \delta_{\mathcal{X}}(x)\\
            \text{where } \delta(x) = 
            \begin{cases} 
            0, &  for x \in \mathcal{X}\\ 
            +\infty, &  o.w. 
            \end{cases}
        \end{aligned}
    \end{align*}
\end{enumerate}

"Rules" for dealing with $\pm \infty$ and $a \in \mathbb{R}$:
\begin{enumerate}
    \item $a+\infty = +\infty + a = +\infty$
    \item $a-\infty = -\infty + a = -\infty$
    \item 
    \begin{align*}
        a \cdot \infty = 
        \begin{cases} 
        \infty, &if \;a > 0  \\ 
        -\infty, &if \;a < 0
        \end{cases}
    \end{align*}
    \item $0 \cdot \pm \infty = 0$
    \item $-\infty < a < \infty \quad \forall a \in \mathbb{R}$
\end{enumerate}

\subsection{Closed Functions}
\begin{definition}
    $epi(f) := \{ (x,y): x \in \mathbb{E}, y \in \mathbb{R}, f(x) \le y \}$
\end{definition}

\begin{definition}
    A function $f: \mathbb{E} \to [-\infty,\infty]$ is said to be closed if $epi(f)$ is closed.
\end{definition}

\begin{proposition}
    For $C \subseteq \mathbb{E}$, $\sigma_{C}(x)$ is closed $\iff C$ is closed.
\end{proposition}
\begin{proof}
    $epi(C) = C \times \mathbb{R}_{+}$
\end{proof}

\begin{remark}
    $f$ is closed $\not\iff$ $dom(f)$ is closed.
\end{remark}

\begin{example}
    \begin{align*}
        f(x) = \begin{cases} 
        \frac{1}{x}, &x > 0  \\ 
        \infty, & x \le 0   
        \end{cases}
    \end{align*}
    Then $dom(f) = (0,\infty)$ is open. And we see that:
    \begin{align*}
        epi(f) = \{ (x,y) \in \mathbb{R}^{2}: x>0, \frac{1}{x} \le y \}
    \end{align*}
\end{example}

\subsection{Related Concepts}
\begin{enumerate}
    \item Lower Semicontinuity:
    \begin{definition}
        $f:\mathbb{E} \to [-\infty,+\infty]$ is l.s.c. at $x \in \mathbb{E}$ if 
        \begin{align*}
            f(x) \le \liminf_{n \to \infty} f(x_n) 
        \end{align*}
        for any sequence $\{ x_n  \}_{n\ge 1} \in \mathbb{E}$ s.t.
        $x_n \to x $ as $n \to \infty$.

        f is said to be l.s.c. if it is l.s.c. at all $x \in \mathbb{E}$.
    \end{definition}
    \item Level set: 
    defined for $\alpha \in \mathbb{R}, \; f:\mathbb{E} \to [-\infty,+\infty]$.
    \begin{align*}
        Lev(f,\alpha) = \{ x \in \mathbb{E}:f(x) \le \alpha \}
    \end{align*}
\end{enumerate}

\begin{theorem}
    If $f:\mathbb{E} \to [-\infty,+\infty]$. Then all of the following statements are equivalent:
    \begin{enumerate}
        \item $f$ is l.s.c.
        \item $f$ is closed.
        \item $Lev(f,\alpha)$ is closed, $\forall \alpha \in \mathbb{R}$
    \end{enumerate}
\end{theorem}

\subsection{Operations preserving closedness}
\begin{enumerate}
    \item If $f: \mathbb{V} \to [-\infty,+\infty]$ is closed, $A:\mathbb{E} \to \mathbb{V}$ is a linear transformation
    and $b \in \mathbb{V}$, then 
    \begin{align*}
        g(x) = f(A(x)+b) \text{ is closed.}
    \end{align*}
    \item If $f_1, \ldots ,f_m: \mathbb{E} \to (-\infty,+\infty]$ are closed and $\alpha_1, \ldots ,\alpha_m \in \mathbb{R}_{+}$,
    then 
    \begin{align*}
        f(x) = \sum_{i=1}^{n}\alpha_{i} f_{i}(x) \text{ is closed}
    \end{align*}
    \item Given an index set $I$ and functions $f_i: \mathbb{E} \to (-\infty,\infty]$, $i \in I$,
    that are closed, the function 
    \begin{align*}
        f(x) = \sup_{i \in I} f_{i}(x) \text{ is closed.}
    \end{align*}
\end{enumerate}

\subsection{Closedness vs Continuity}
Bottom line: If $f$ has closed domain + continuous over the domain $\Longrightarrow$ closed.

But closed $\not\iff$ continuous over the domain.

\begin{theorem}
    Let $f: \mathbb{E} \to (-\infty,+\infty]$ be continuous over its domain and suppose $dom(f)$ is closed
    $\Longrightarrow$ f is closed.
\end{theorem}

\begin{proof}
    Argue that $epi(f)$ is closed.

    Take any sequence  $\{ (x_n,y_n) \}_{n\ge 1} \in epi(f)$ that converges to some $(x_{*}, y_{*})$ as $n \longrightarrow \infty$

    To argue: $(x_{*}, y_{*}) \in epi(f)$: we know that $x_n \in dom(f)$, $x_n \longrightarrow x_{*}$,
    $dom(f)$ is closed $\Longrightarrow x_{*} \in dom(f)$

    By the definition of $epi(f)$:
    \begin{align*}
        f(x_n) \le y_n
    \end{align*}
    Since f is continuous over $dom(f)$ and $\{ x_n \}_{n}, x_{*} \in dom(f)$ we can take the limit $n \longrightarrow \infty$
    \begin{align*}
        f(x_{*}) \le y_{*}\\
        \Longrightarrow (x_{*},y_{*}) \in epi(f)
    \end{align*}
\end{proof}

\begin{example}[closed $\not\Longrightarrow$ continuous on its domain]
 \item 
 \begin{align}
    f_{\alpha}(x) = \begin{cases} 
    \alpha, &  x=0\\ 
    x, &   0 < x \le 1\\
    \infty, & elsewhere
    \end{cases}
 \end{align}
 When $\alpha < 0$, then it's l.s.c., i.e., closed, but it's not continuous.
 \item $l_{0}$ "norm"
 \begin{align*}
    f(x) = \| \vx \| _{0} = |\{ i:\vx_{i} \neq 0 \}| 
 \end{align*}
 $f$ is not continuous but it's closed.

 \begin{align*}
    f(x) = \sum_{i=1}^{d}I(\vx_{i})
 \end{align*}
 where 
 \begin{align*}
    I(y) = \begin{cases} 
    0, & y=0 \\ 
    1, & y\neq 0  
    \end{cases}
 \end{align*}
 We know 
 \begin{align*}
    Lev(I,\alpha) = \begin{cases} 
    \emptyset , &  \alpha < 0\\ 
    \{ 0 \}, &   0 \le \alpha < 1\\
    \mathbb{R}, & \alpha \ge 1
    \end{cases}
 \end{align*}
 Then $I$ is closed. $\Longrightarrow$ the sum of them is closed.
\end{example}










\end{document}